\documentclass[12pt,                    % corpo del font principale
               a4paper,                 % carta A4
               %twoside,                 % impagina per fronte-retro
               %openright,               % inizio capitoli a destra
               english,
               italian,
               ]{article}

%**************************************************************
% Importazione package
%**************************************************************

\usepackage{amsmath,amssymb,amsthm}    % matematica

\usepackage[english, italian]{babel}    % per scrivere in italiano e in inglese;
                                        % l'ultima lingua (l'italiano) risulta predefinita
\usepackage[top=1.5in,bottom=1in,right=1in,left=1in,headheight=65pt,headsep=1cm]{geometry}
%\usepackage[paperwidth=5.5in, paperheight=8.5in]{geometry}


\usepackage{float}

\usepackage{caption}                    % didascalie

\usepackage{chngpage,calc}              % centra il frontespizio

\usepackage{csquotes}                   % gestisce automaticamente i caratteri (")

\usepackage{emptypage}                  % pagine vuote senza testatina e piede di pagina

\usepackage{epigraph}                   % per epigrafi
\usepackage[bottom]{footmisc}           % put footnotes at the bottom
\usepackage{eurosym}                    % simbolo dell'euro
\usepackage{enumitem}        % separation between list items
\usepackage{ifxetex}                    %compilazione xelatex

\ifxetex

\RequirePackage{xltxtra,fontspec,xunicode}

\RequirePackage{unicode-math}

\else
  \RequirePackage[utf8]{inputenc}

\fi


\usepackage{verbatim}

\usepackage{graphicx}                   % immagini

\usepackage[
    colorlinks=true,       % false: boxed links; true: colored links
    linkcolor=cyan,          % color of internal links (change box color with linkbordercolor)
    citecolor=cyan,        % color of links to bibliography
    filecolor=magenta,      % color of file links
    urlcolor=cyan,          % color of external links
]{hyperref}                   % collegamenti ipertestuali


\usepackage{longtable} % per le tabelle multipagina
\usepackage{listings}                   % codici

\usepackage{microtype}                  % microtipografia

\usepackage{mparhack,fixltx2e,relsize}  % finezze tipografiche

\usepackage{nameref}                    % visualizza nome dei riferimenti

\usepackage[font=small]{quoting}        % citazioni

\usepackage{subcaption}                     % sottofigure, sottotabelle

\usepackage[section]{placeins} %\FloatBarrier

\usepackage[italian]{varioref}          % riferimenti completi della pagina

\usepackage[dvipsnames]{xcolor}         % colori

\usepackage{booktabs}                   % tabelle
\usepackage{tabularx}                   % tabelle di larghezza prefissata
\usepackage{longtable}                  % tabelle su più pagine
\usepackage{indentfirst}
\usepackage{ltxtable}                   % tabelle su più pagine e adattabili in larghezza
\usepackage[toc, acronym]{glossaries}   % glossario
                                        % per includerlo nel documento bisogna:
                                        % 1. compilare una prima volta tesi.tex;
                                        % 2. eseguire: makeindex -s tesi.ist -t tesi.glg -o tesi.gls tesi.glo
                                        % 3. eseguire: makeindex -s tesi.ist -t tesi.alg -o tesi.acr tesi.acn
                                        % 4. compilare due volte tesi.tex.

\usepackage[backend=bibtex,style=ieee,hyperref,backref]{biblatex}
                                        % eccellente pacchetto per la bibliografia;
                                        % produce uno stile di citazione autore-anno;
                                        % lo stile "numeric-comp" produce riferimenti numerici
                                        % per includerlo nel documento bisogna:
                                        % 1. compilare una prima volta tesi.tex;
                                        % 2. eseguire: biber tesi
                                        % 3. compilare ancora tesi.tex.


\usepackage[linesnumbered,ruled,vlined]{algorithm2e}% Use the algorithmicx package for pseudocode
%\usepackage{algorithm}
%\usepackage{algpseudocode}

%verbatim
\usepackage{fancyvrb}

\usepackage{fancyhdr}
\pagestyle{fancy}

%\rfoot{Pagina: \emph{ \thepage\ / \pageref{LastPage}}}

\title{
Assignment 1 \\
Internet Programming Course
}
\author{
Stefano Sandonà: ssa223 \\
Federico Ziliotto: fzo300
}
\date{\today}


%********        Headers e footer       ***********************%

%\fancyhead[R]{\small\rightmark}
%\fancyhead[L]{\small\leftmark}
%\fancyfoot[C]{\small\thepage}
%\fancyhead[LO]{\small\leftmark}% odd page header and number to right top
%\fancyhead[RE]{\small\leftmark}%Even page header and number at left top
%\fancyfoot[L,R,C]{}
%\renewcommand{\headrulewidth}{0pt}% disable the underline of the header part


% \lhead{\textsc{\title}}
% \chead{}
% \rhead{\rightmark}
% %\lfoot{\thepage}
% \cfoot{\thepage}
% %\rfoot{\textsc{\leftmark}}


%line spacing
\linespread{1.2}
\setlist{nosep}
\setlength{\parindent}{14pt}   % larghezza rientro della prima riga
%\setlength{\parskip}{10pt}   % distanza tra i paragrafi


%%--------------------------------------------------------------------------
%% Font settings
%%--------------------------------------------------------------------------
\ifxetex
\setromanfont{Droid Serif}
\setsansfont{Droid Sans}
\setmonofont{Droid Sans Mono}
%\setmathfont{Cambria Math}
\else
  \usepackage[T1]{fontenc}
  \usepackage{newtxtext,newtxmath}
%Font packages
%\usepackage{times}
 %\usepackage{lmodern}
 %\usepackage{mathpazo}
 %\usepackage{kpfonts}
 %\usepackage{mathptmx}
 %\usepackage{times,mtpro2}
 %\usepackage{stix}
 %\usepackage{txfonts}
 %\usepackage{newtxtext,newtxmath}
 %\usepackage{libertine} \usepackage[libertine]{newtxmath}
%\usepackage{newtxtext,newtxmath}
\fi


%%--------------------------------------------------------------------------
%% Algorithms settings
%%--------------------------------------------------------------------------
\newcommand\mycommfont[1]{\scriptsize\ttfamily\textcolor{blue}{#1}}
\SetCommentSty{mycommfont}

\makeatletter
\renewcommand{\listalgorithmcfname}{Lista degli Algoritmi}%
\renewcommand{\algorithmcfname}{Algoritmo}%
\renewcommand{\algocf@typo}{}%
\renewcommand{\@algocf@procname}{Procedura}
\renewcommand{\@algocf@funcname}{Funzione}
\makeatother

\floatstyle{ruled}
\newfloat{program}{thbp}{lop}
\floatname{program}{Programma}

\floatstyle{ruled}
\newfloat{algorithm}{thb}{lop}
\floatname{algorithm}{Algoritmo}


%**********   Hyperlink setup ************%
\hypersetup{
    bookmarks=true,         % show bookmarks bar?
    pdftoolbar=true,        % show Acrobat’s toolbar?
    pdfmenubar=true,        % show Acrobat’s menu?
    pdffitwindow=false,     % window fit to page when opened
    pdfstartview={FitH},    % fits the width of the page to the window
    pdftitle={},    % title
    pdfauthor={},     % author
    pdfsubject={},   % subject of the document
    %pdfcreator={Federico Ziliotto},   % creator of the document
    %pdfproducer={Producer}, % producer of the document
    pdfkeywords={}, % list of keywords
    pdfnewwindow=true,      % links in new PDF window
    colorlinks=true,       % false: boxed links; true: colored links
    linkcolor=cyan,          % color of internal links (change box color with linkbordercolor)
    citecolor=cyan,        % color of links to bibliography
    filecolor=magenta,      % color of file links
    urlcolor=cyan           % color of external links
}
%**************************************************************
% Impostazioni di caption
%**************************************************************
\captionsetup{
    tableposition=top,
    figureposition=bottom,
    font=small,
    format=hang,
    labelfont=bf
}



%**************************************************************
% Impostazioni di graphicx
%**************************************************************
\graphicspath{{Images/}} % cartella dove sono riposte le immagini


%**************************************************************
% Impostazioni di itemize
%**************************************************************
%\renewcommand{\labelitemi}{$\ast$}

\renewcommand{\labelitemi}{$\bullet$}
%\renewcommand{\labelitemii}{$\cdot$}
%\renewcommand{\labelitemiii}{$\diamond$}
%\renewcommand{\labelitemiv}{$\ast$}


%**************************************************************
% Impostazioni di listings
%**************************************************************
\lstset{
    language=[LaTeX]Tex,%C++,
    keywordstyle=\color{RoyalBlue}, %\bfseries,
    basicstyle=\small\ttfamily,
    %identifierstyle=\color{NavyBlue},
    commentstyle=\color{Green}\ttfamily,
    stringstyle=\rmfamily,
    numbers=none, %left,%
    numberstyle=\scriptsize, %\tiny
    stepnumber=5,
    numbersep=8pt,
    showstringspaces=false,
    breaklines=true,
    frameround=ftff,
    frame=single
}


%**************************************************************
% Impostazioni di biblatex
%**************************************************************
%\bibliographystyle{unsrtnat}

\bibliography{bibliography} % database di biblatex
%\bibliographystyle{alpha}
%\addbibresource{bibliography.bib}



%%%%%%%%%%%%%%%%% END OF PREAMBLE %%%%%%%%%%%%%%%%

\begin{document}
\maketitle

%\include{Sections/Introduzione.tex}
%\include{Section2}
%\include{Section3}
%\include{Section4}
%\include{Conclusion}

%*******    Figure and Subfigure example ***********
%\begin{figure}[tbh]
% \includegraphics[width=1\linewidth]{circle}
% \caption[Circonferenza]{Circle}
% \label{fig:circle}
% \end{figure}%

% \begin{figure}[tbh]
% \begin{subfigure}{.5\textwidth}
% \includegraphics[width=1\linewidth]{circle}
% \caption[Circonferenza]{Circle}
% \label{fig:circle}
% \end{subfigure}%
% \begin{subfigure}{0.5\textwidth}
% \includegraphics[width=1\linewidth]{random}
% \caption[Random]{Random}
% \label{fig:random}
% \end{subfigure}
% \end{figure}
%%%***************************************%%

\section{Answers to the Questions}
\textbf{Q-A}: How many processes must your shell create when receiving a piped command? How many pipes? Until when must the shell wait to accept another command? \newline

When receiving a commands in pipe, the shell program must create a number of processes equal of the number of pipe symbols found plus one (e.g. cmd1 | cmd2 | cmd3, we have three commands and two pipe symbols). Each command so is managed by one process. \newline
In addition the shell must create one pipe for each pair of commands, so one pipe for each pipe symbol. This is due to the fact that the output of one command has to be redirected in the input of the following command, so one pipe for each pair of commands.\newline
The shell will have to wait the termination of all the child processes that are executing the commands in pipe. 
The processes created to execute the piped commands share the same file descriptors, in particular those for the standard input and output. If the shell were to accept and execute other processes when the ones in the pipe are still executing, the access to the standard input and output would be shared by processes that are doing different things, that may lead to wrong or undesirable results. \newline

\textbf{Q-B}:Can you implement a shell program which only utilizes threads (instead of processes)? If your answer is yes, then write a thread-based version of \texttt{mysh1}, call it \texttt{mysh1\_th}, and include it in your submission. If your answer is no, explain why. \newline

No, it is not possible. To execute shell commands we need to call the ''exec'' function, so we can run programs on ''/bin'' directory. The ''exec'' command, change the code of the current process with another one specified. So, calling an ''exec'' inside a thread cause the change of the code of the entire program, for these reason each command of a shell must be executed by another process. Calling ''exec'' without creating a new process means that the shell program is substituted with the code of the specified command and this is not what we need, because when the command execution is finished we cannot continue to accept other command (the shell code was substituted). \newline


\textbf{Q-C}: Can you use the \texttt{cd} command with shell \texttt{mysh3}? Why?\newline

Yes, we can implement the shell to use it. The ''cd'' is not an executable to call but it is implemented inside the shell because it maintains the working directory where shell commands can find files and programs. 
In mysh3 we can implement ''cd'' to change the current working directory of the shell process when it is the only command present (by invoking \texttt{chdir}). If the command is in pipe with other commands, we have to take the precaution to wait for the first command execution to finish before invoking \texttt{execvp} on the second command, because the first one could have been a ''cd'' that changed the directory, the second one should work on that directory. We could also not allow the ''cd'' command in pipes, then it would be safe to run the two process that executes the commands concurrently (because pipes are blocking so if one process needs to wait for the result of the other it can do it by reading the shared pipe).


\section{Programs Descriptions}
\subsection{\texttt{mysh1}}
This is the simple implementation of a shell that receives one command at a time (without arguments) and executes it. The source of the command is the standard input (\texttt{stdin}), which is the keyboard input stream. The \texttt{getline} function provides the entire content of one line of input from that stream (so it waits to return until the newline character is inserted). The content of the line is stored in the first argument that is passed, while the second one will contain the size of the buffer that is used to store the line. \newline

The call to \texttt{strtok} searches for the start of the command (it skips characters like spaces and tabs) and returns a pointer to that. If no command is found (the newline is inserted before any symbol that may be the start of a command) then the shell will ask for another input line. \newline

If a command is found, first it's compared to the \texttt{exit} directive (to exit the whole program), then a new process is started with the \texttt{fork()} system call to execute the program with the name of the command. The call to \texttt{execvp()} takes the name of the command to execute and searches it in the directories listed in the \textbf{PATH} environment variable.
When the child process has finished executing (or there was an error when calling \texttt{execvp}), it terminates. In the meantime, the parent process waits for the termination of any of its children with the invocation of \texttt{wait(NULL)}. In this case there is only one child to wait. After that, the parent return at the top of the cycle and waits for another command.
\subsection{\texttt{mysh2}}
To extend \texttt{mysh1} to accept arguments for the invoked commands, we used an array of pointers to the arguments. These pointer point towards the input string and are parsed using \texttt{strtok} to eliminate characters like spaces and tabs between each argument. \newline

For the sake of simplicity and correctness, we allocate each time new memory to use to store the array of pointers (we use dynamic allocation because we don't know the number of arguments that may be used) and we free the memory at every new command read. \\ 

\texttt{mysh2} makes use of these two functions:
\begin{itemize}
\item{\texttt{char **parseArguments(char *input, char *delim, int *size)}}: takes the string in \texttt{input} and a character array of delimiters and returns an array of pointers to tokens that are obtained from the initial input by separating the tokens with the \texttt{delim}'s characters. The number of tokens is saved in the variable \texttt{size} and the function return the array of pointers to the tokens.
\item{\texttt{void deallocation(char **array, int size)}}: this function takes an array of pointers to memory locations that were previously allocated with \texttt{malloc} and deallocates them by invoking \texttt{free} for each pointer.
\end{itemize}

After the call to the function \texttt{parseArguments} has returned, in the array \texttt{args} we have:
\begin{itemize}
\item \texttt{args[0]}: contains the command to execute (the first token found);
\item \texttt{args[1]...args[size-1]}: have pointers to the arguments of the command;
\item \texttt{args[size]}: this is set to be NULL, since the call to \texttt{malloc} and \texttt{realloc} does not initialize the value of the new memory locations.
\end{itemize}

So the array \texttt{args} contains the command and the arguments for it, as required by the system call \texttt{execvp}, that is called by the child process to execute the command.

\subsection{\texttt{mysh3}}
To extend \texttt{mysh2} to accept piped commands, we first used an array of pointers to the commands and than for each command we used an array of pointers to its arguments. To create the first array we separate strings using the pipe symbol as delimiter (using the function \texttt{strtok}), then to create an array of arguments for each command, using spaces and tabs as delimiters.  \newline

This version of the shell accepts only two commands separated by a pipe or just one standard command. To know in which case we are is sufficient to check the size of the array of commands. There could be the possibility that a user insert a piped command (insert the pipe symbol) without one of the two needed commands (e.g. ''|ls'' or ''ls|''). A check in for this case is performed and eventually a message ''no command before or after pipe'' is shown.\newline

If two commands are detected on the string inserted by the user, two new processes are created. In addition a pipe is created. The stdout of the first process is redirected to the writing side of the pipe, while the stdin of the second process is redirected to the reading side of the pipe. Following this way, the output of the first command becomes the input for the second one. \newline

\texttt{mysh3} makes use of the same function that \texttt{mysh2} used:
\begin{itemize}
\item{\texttt{char **parseArguments(char *input, char *delim, int *size)}}
\item{\texttt{void deallocation(char **array, int size)}}
\end{itemize}
to create the arrays (\texttt{args}, \texttt{args1} and \texttt{args2}) that contain the commands and their arguments, that are then used to as arguments for the invocation of \texttt{execvp}.



\subsection{\texttt{syn1}}
This program creates two processes that competes to access the same output stream. To avoid that the messages they want to print overlaps, we use one semaphore. Every time one of the processes wants to use the \texttt{display} function, the try to perform a down operation on the semaphore (to avoid the other process doing the same) and when they are finished they perform an up operation freeing it. If the semaphore is already take, the system call \texttt{semop} will block the execution of the process until an up operation is performed on that semaphore.
\subsection{\texttt{syn2}}
In this program one process can initially start printing ''ab'', then it alternates execution with the process that prints ''cd''. In this case we use two semaphores: one that allows to print ''ab'' while the other will allow to print ''cd\textbackslash{}n''. The initial state of the semaphores will be ''up'' for the first one while down for the other and the sequence of execution will be:
\begin{enumerate}
\item abSem Up, cdSem Down
\item abSem Down, cdSem Down
\item print ''ab''
\item abSem Down, cdSem Up
\item abSem Down, cdSem Down
\item print ''cd\textbackslash{}n''
\item abSem Up, cdSem Down
\end{enumerate}
Then the sequence of execution repeats.
\subsection{\texttt{syn1.java}}
This program creates two threads that invoke the same function to display a message. Without mutual exclusion the output of the two programs could overlap with each other. Since the lock Object is the same for the two threads (because it is a static variable of the class) and the code in the synchronized block can be executed only by the thread that has gained possession of the lock, we are guaranteed that the outputs of the threads will not be mixed together.
\subsection{\texttt{syn2.java}}
In this case we use a program structure similar to the problem producer/consumer (we can think of the program who prints ''ab'' as the producer and the other as the consumer, with a buffer of size one since everything that is produced needs to be consumed immediately). \newline
The class \texttt{PrintMonitor} contains both the condition variable (we just need a boolean to allow either one of the threads to access) and the synchronized methods \texttt{printAb} and \texttt{printCd}. The methods needs to be synchronized because the access to the condition variable has to be done in mutual exclusion. \newline
Both threads just need to invoke their respective methods on the shared monitor object. The code of the methods is symmetrical: if the condition variable is false (true for the other), the thread waits (on the monitor object), when it is woken up by a notify, it checks again the condition variable and proceeds if it has the right value. After printing, it changes the value of the condition variable and notify the other thread, waking it up if it was waiting for a change in the value of that variable. In this way we are sure that the two threads that are running concurrently will alternate, because only one at a time can enter in the synchronized block and because each one has to wait for the other to change the value of the condition variable before advancing.
\subsection{\texttt{synthread1}}
In this program two pthreads are created that have the same code (the function \texttt{print} but called with different arguments). The two do the same thing, they try to invoke the \texttt{display} function to write on the standard output. Since the two threads run concurrently, they could interfere with each other. To avoid this we use the primitives \texttt{pthread\_mutex\_lock} and \texttt{pthread\_mutex\_unlock} that assure us that only one thread at a time can invoke \texttt{display} (because the code for \texttt{print} is the same for both and the mutex is a global variable that is shared by all the threads born from the same process). At the end we wait for the threads to terminate and destroy the mutex.
\subsection{\texttt{synthread2}}
This program has the same structure of \texttt{syn2.java} where the thread that prints ''ab''(\texttt{printAb}) and the one that prints ''cd\textbackslash{}n'' (\texttt{printCd}) alternates using mutexes and condition variables. The condition variable \texttt{write\_ab} is checked to be \texttt{true} for the first one and \texttt{false} for the second. If the condition is not satisfied the pthread waits until it is notified of a change of the variable from the other pthread. When access in the critical region is gained each pthread does its printing job, change the value of the condition variable to permit access to the other and wakes up it with a \texttt{pthread\_cond\_signal}, and, finally, release the mutex.

\clearpage

%\bibbycategory % equivale a dare un \printbibliography per ogni categoria

\end{document}
