\documentclass[12pt,                    % corpo del font principale
               a4paper,                 % carta A4
               %twoside,                 % impagina per fronte-retro
               %openright,               % inizio capitoli a destra
               english,
               italian,
               ]{article}

%**************************************************************
% Importazione package
%**************************************************************

\usepackage{amsmath,amssymb,amsthm}    % matematica

\usepackage[english, italian]{babel}    % per scrivere in italiano e in inglese;
                                        % l'ultima lingua (l'italiano) risulta predefinita
\usepackage[top=1.5in,bottom=1in,right=1in,left=1in,headheight=65pt,headsep=1cm]{geometry}
%\usepackage[paperwidth=5.5in, paperheight=8.5in]{geometry}


\usepackage{float}

\usepackage{caption}                    % didascalie

\usepackage{chngpage,calc}              % centra il frontespizio

\usepackage{csquotes}                   % gestisce automaticamente i caratteri (")

\usepackage{emptypage}                  % pagine vuote senza testatina e piede di pagina

\usepackage{epigraph}                   % per epigrafi
\usepackage[bottom]{footmisc}           % put footnotes at the bottom
\usepackage{eurosym}                    % simbolo dell'euro
\usepackage{enumitem}        % separation between list items
\usepackage{ifxetex}                    %compilazione xelatex

\ifxetex

\RequirePackage{xltxtra,fontspec,xunicode}

\RequirePackage{unicode-math}

\else
  \RequirePackage[utf8]{inputenc}

\fi


\usepackage{verbatim}

\usepackage{graphicx}                   % immagini

\usepackage[
    colorlinks=true,       % false: boxed links; true: colored links
    linkcolor=cyan,          % color of internal links (change box color with linkbordercolor)
    citecolor=cyan,        % color of links to bibliography
    filecolor=magenta,      % color of file links
    urlcolor=cyan,          % color of external links
]{hyperref}                   % collegamenti ipertestuali


\usepackage{longtable} % per le tabelle multipagina
\usepackage{listings}                   % codici

\usepackage{microtype}                  % microtipografia

\usepackage{mparhack,fixltx2e,relsize}  % finezze tipografiche

\usepackage{nameref}                    % visualizza nome dei riferimenti

\usepackage[font=small]{quoting}        % citazioni

\usepackage{subcaption}                     % sottofigure, sottotabelle

\usepackage[section]{placeins} %\FloatBarrier

\usepackage[italian]{varioref}          % riferimenti completi della pagina

\usepackage[dvipsnames]{xcolor}         % colori

\usepackage{booktabs}                   % tabelle
\usepackage{tabularx}                   % tabelle di larghezza prefissata
\usepackage{longtable}                  % tabelle su più pagine
\usepackage{indentfirst}
\usepackage{ltxtable}                   % tabelle su più pagine e adattabili in larghezza
\usepackage[toc, acronym]{glossaries}   % glossario
                                        % per includerlo nel documento bisogna:
                                        % 1. compilare una prima volta tesi.tex;
                                        % 2. eseguire: makeindex -s tesi.ist -t tesi.glg -o tesi.gls tesi.glo
                                        % 3. eseguire: makeindex -s tesi.ist -t tesi.alg -o tesi.acr tesi.acn
                                        % 4. compilare due volte tesi.tex.

\usepackage[backend=bibtex,style=ieee,hyperref,backref]{biblatex}
                                        % eccellente pacchetto per la bibliografia;
                                        % produce uno stile di citazione autore-anno;
                                        % lo stile "numeric-comp" produce riferimenti numerici
                                        % per includerlo nel documento bisogna:
                                        % 1. compilare una prima volta tesi.tex;
                                        % 2. eseguire: biber tesi
                                        % 3. compilare ancora tesi.tex.


\usepackage[linesnumbered,ruled,vlined]{algorithm2e}% Use the algorithmicx package for pseudocode
%\usepackage{algorithm}
%\usepackage{algpseudocode}

%verbatim
\usepackage{fancyvrb}

\usepackage{fancyhdr}
\pagestyle{fancy}

%\rfoot{Pagina: \emph{ \thepage\ / \pageref{LastPage}}}
